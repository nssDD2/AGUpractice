%% -*- coding: utf-8 -*-
\documentclass[12pt,a4paper]{scrartcl} 
\usepackage[utf8]{inputenc}
\usepackage[english,russian]{babel}
\usepackage{indentfirst}
\usepackage{misccorr}
\usepackage{graphicx}
\usepackage{amsmath}
\begin{document}

% Что должно быть во введении
\begin{enumerate}
\end{enumerate}
\large\tableofcontents

\vspace{25\baselineskip}

\section{Теория}
\subsection{Техническое задание}
\textbf {Задание:}

Создать программу для генерации случайных паролей заданной длины и сложности.

\subsection{Теоретическая часть}

Пароль создается при помощи модуля random – Модуль python random помогает пользователю генерировать псевдослучайные числа. Внутри модуля есть различные функции, которые просто зависят от функции “random()”. Эта функция генерирует случайное число с плавающей запятой равномерно в полуоткрытом диапазоне [0.0, 1.0), т.е. Она генерирует десятичное число, большее или равное 0 и строго меньшее единицы. Другие функции используют это число по-своему. Эти функции можно использовать для байтов, целых чисел и последовательностей. для нашей задачи нас интересуют последовательности. Существуют функции random . варианты, которые принимают последовательность в качестве аргумента и возвращают случайный элемент из этой последовательности.

1.length указывается длина пароля.

2.alphabets указывается количество символов содержащихся в пароле.

3.digits указывается количество цифр содержащихся в пароле

4.special characters  указывается количество специальных символов содержащихся в пароле

\section{Ход работы}
\label{sec:exp}

\subsection{Код приложения}
\label{sec:exp:code}
\begin{verbatim}
import string
import random
alphabets = list(string.ascii_letters)
digits = list(string.digits)
special_characters = list("!@#$%^&*()")
characters = list(string.ascii_letters + string.digits + "!@#$%^&*()")
def generate_random_password():

	length = int(input("Укажите длину: "))
 
	alphabets_count = int(input("Ввод символов в пароле: "))
 
	digits_count = int(input("Ввод цифр в пароле: "))
 
	special_characters_count = int(input("Ввод специальных символов в пароле: "))
 
	characters_count = alphabets_count + digits_count + special_characters_count
	if characters_count > length:
 
		print("Общее количество символов превышает длину пароля.")
		return
	password = []
	for i in range(alphabets_count):
		password.append(random.choice(alphabets))
	for i in range(digits_count):
		password.append(random.choice(digits))
	for i in range(special_characters_count):
		password.append(random.choice(special_characters))
	if characters_count < length:
		random.shuffle(characters)
		for i in range(length - characters_count):
			password.append(random.choice(characters))
	random.shuffle(password)
	print("".join(password))
generate_random_password()
        \end{verbatim}


\subsection{Работа программы} 



\begin{wrapfigure}
  \begin{center}
    \includegraphics[width=0.5\textwidth]{workscr.png}
  \end{center}
  \caption{Рис.1  Пример работы программы.}\label{fig:ex}
\end{wrapfigure}

\newpage
\begin{thebibliography}{9}
\bibitem{Knuth-2003}Кнут Д.Э. Всё про \TeX. \newblock --- Москва: Изд. Вильямс, 2003 г. 550~с.
\bibitem{Lvovsky-2003}Львовский С.М. Набор и верстка в системе \LaTeX{}. \newblock --- 3-е издание, исправленное и дополненное, 2003 г.
\bibitem{Voroncov-2005}Воронцов К.В. \LaTeX{} в примерах. 2005 г.
\end{thebibliography}

\end{document}